\documentclass[12pt,letterpaper]{article}
\usepackage{fullpage}
\usepackage[top=2cm, bottom=4.5cm, left=2.5cm, right=2.5cm]{geometry}
\usepackage{amsmath,amsthm,amsfonts,amssymb,amscd}
\usepackage{lastpage}
\usepackage{enumerate}
\usepackage{fancyhdr}
\usepackage{mathrsfs}
\usepackage{xcolor}
\usepackage{graphicx}
\usepackage{listings}
\usepackage{hyperref}
\usepackage[russian]{babel}

\hypersetup{%
  colorlinks=true,
  linkcolor=blue,
  linkbordercolor={0 0 1}
}
 
\renewcommand\lstlistingname{Algorithm}
\renewcommand\lstlistlistingname{Algorithms}
\def\lstlistingautorefname{Alg.}

\lstdefinestyle{Python}{
    language        = Python,
    frame           = lines, 
    basicstyle      = \footnotesize,
    keywordstyle    = \color{blue},
    stringstyle     = \color{green},
    commentstyle    = \color{red}\ttfamily
}

\setlength{\parindent}{0.0in}
\setlength{\parskip}{0.05in}

% Edit these as appropriate
\newcommand\hwnumber{1}                  % <-- homework number
\newcommand\NetIDa{Rudenko Varvara}           % <-- NetID of person #1


\pagestyle{fancyplain}
\headheight 35pt
\lhead{\NetIDa}
\chead{\textbf{\Large Homework \hwnumber}}
\rhead{ \today}
\lfoot{}
\cfoot{}
\rfoot{\small\thepage}
\headsep 1.5em

\begin{document}
\section{General optimization problems}

\subsection*{Задача №1}
$\textbf{Give an explicit solution of the following LP.}$
$$ 
\begin{aligned}
&c^Tx\rightarrow min_{x\in \mathbb{R}^n}\\ 
&\text{s.t.}\ Ax=b
\end{aligned}
$$
Лагранжиан этой задачи:
$$ L(x,\lambda)=c^Tx+\lambda^T(Ax-b)$$
Задача выпуклая (линейная, ограничения афинные).
Условия Каруша-Куна-Таккера:
$$ 
\begin{aligned}
&\bigtriangledown_x L: c+A^T\lambda=0\\ 
&\bigtriangledown_\lambda L: Ax=b
\end{aligned}
$$
Тогда можно говорить о двух случаях для $Ax=b$:
\begin{itemize}
	\item[1. ] Нет решения => бюджетное множество пустое, оптимальное значение $p^*=\infty$
	\item[2. ] Решение есть. Можно записать его через псевдообратную матрицу- $x^*=A^\dagger b$, выполнены условия Слейтера => по ККТ получили оптимальное решение $x^*$ и значение $p^*=c^T A^\dagger b$
\end{itemize}



\subsection*{Задача №2}
$\textbf{Give an explicit solution of the following LP.}$
$$ 
\begin{aligned}
&c^Tx\rightarrow min_{x\in \mathbb{R}^n}\\ 
&\text{s.t.}\ 1^Tx=1\\
&x\succcurlyeq0
\end{aligned}
$$
Лагранжиан этой задачи:
$$ L(x,\lambda,\mu)=c^T x+\lambda(1^T x-1)-\mu^T x  $$
Задача выпуклая, так как линейная, условия линейные, выполняются условия Слейтера (существует допустимая точка, все компоненты $x =\frac{1}{n}$). Тогда получаем достаточные ККТ:
$$
\begin{aligned}
&\bigtriangledown_x L: c+\lambda\times1^T-\mu=0\\ 
&\bigtriangledown_\lambda L: 1^Tx=1\\
&\mu_j\geq,\ j=[1,n]\\
&\mu_j x_j=0,\ j=[1,n]\\
&x\succeq0
\end{aligned}
$$ 
Выбираем вектор c, где на месте минимальной компоненты $c_i$ у $x^*=(0,0,...,1,0,...,0)$ стоит 1. => $\lambda^*=-c_i,\ \mu^*_i=0, \mu_j=c_j-c_i\geq0$.\\
$x^*, \mu^*, \lambda^*$ - решение системы, так как выполнено ККТ, $x^*$ - решение задачи => оптимальное значение $p^*=c_i$


\subsection*{Задача №3}
$\textbf{Give an explicit solution of the following LP.}$
$$ 
\begin{aligned}
&c^Tx\rightarrow min_{x\in \mathbb{R}^n}\\ 
&\text{s.t.}\ 1^Tx=\alpha\\
&0\curlyeqprec x\curlyeqprec1
\end{aligned}
$$


\subsection*{Задача №4}
$\textbf{Give an explicit solution of the following QP.}$
$$ 
\begin{aligned}
&c^Tx\rightarrow min_{x\in \mathbb{R}^n}\\ 
&x^TAx\leqslant1
\end{aligned}
$$

\subsection*{Задача №5}
\subsection*{Задача №6}
\subsection*{Задача №7}
\subsection*{Задача №8}
\subsection*{Задача №9}

\section{Duality}
\subsection*{Задача №1}
\subsection*{Задача №2}
\subsection*{Задача №3}
\subsection*{Задача №4}


\end{document}